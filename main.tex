% Nejprve uvedeme tridu dokumentu s volbami
\documentclass[czech,bachelor,public,dept460,male,oneside]{diploma}
% Dalsi doplnujici baliky maker
\usepackage{subfig}		% makra pro "podobrazky" a "podtabulky"


% Zadame pozadovane vstupy pro generovani titulnich stran.
\ThesisAuthor{Radek Svoboda}

\CzechThesisTitle{Nástroj pro modelování relační databáze}

\EnglishThesisTitle{Relational Database Modeling Tool}

\SubmissionDate{28. dubna 2017}

% Pokud nechceme nikomu dekovat makro zapoznamkujeme.
\Thanks{Rád bych na tomto místě poděkoval vedoucímu mé bakalářské práce Ing. Petru Lukášovi za poskytnuté rady a čas strávený konzultacemi.}

% Zadame cestu a jmeno souboru ci nekolika souboru s digitalizovanou podobou zadani prace.
% Pokud toto makro zapoznamkujeme sazi se stranka s upozornenim.
\ThesisAssignmentImagePath{Figures/Assignment}

% Zadame soubor s digitalizovanou podobou prohlaseni autora zaverecne prace.
% Pokud toto makro zapoznamkujeme sazi se cisty text prohlaseni.

%\AuthorDeclarationImageFile{Figures/AuthorDeclaration.jpg}


\CzechAbstract{Tato bakalářská práce se zabývá návrhem a implementací softwarového nástroje pro návrh schématu relační databáze formou E-R diagramu. Nástroj umožní uživateli nejen tvorbu nových schémat databáze, ale také vizualizaci již existujících schémat a jejich následné modifikace prostřednictvím změn E-R diagramu v grafickém editoru. V první části se práce zabývá notací E-R diagramu a srovnáním již existujících nástrojů určených k tvorbě E-R diagramů. Druhá část se zaměřuje na návrh a implementaci jak grafického editoru, tak na problematiku synchronizace E-R diagramu s existujícím relačním schématem.}

\CzechKeywords{relační databáze, E-R diagram, relační schéma, CASE nástroj}

\EnglishAbstract{This thesis describes design and implementation of software tool for relational database schema design in form of an E-R diagram. Tool is not limited only for creation of new database schemas, but it also provides visualization of already existing schemas and their subsequent modifications through changing E-R diagram in graphic editor. The first part deals with E-R diagram notation and the comparison of existing tools for the creation of E-R diagrams. The second part focuses on the design and implementation of both the graphical editor, and on the synchronization of E-R diagram with existing relational schema.}

\EnglishKeywords{relational database, E-R diagram, relational schema, CASE tool}

\AddAcronym{SQL}{Structured Query Language}
\AddAcronym{DDL}{Data Definition Language}
\AddAcronym{CASE}{Computer-Aided Software Engineering}
\AddAcronym{ERD}{Entity-Relationship Diagram}
\AddAcronym{DBMS}{Database Management System}


% Zacatek dokumentu
\begin{document}

% Nechame vysazet titulni strany.
\MakeTitlePages

% Pokud mame v zaverecne praci vypisy kodu, jinak odstranit.
\lstlistoflistings

% A nasleduje text zaverecne prace.
\section{Úvod}
Strukturované uložení dat je dnes jednou ze základních potřeb většiny vyvíjených aplikací. S rostoucím množstvím uchovávaných dat a potřebou jejich analýzy už dávno není dostačující prosté uložení do souboru. Na tyto potřeby reagují, dnes ve velké míře používané, relační databáze. Před samotným uložením dat je nutné definovat jejich strukturu, konkrétně relační model. Dnešní svět se velmi rychle vyvíjí, tedy mění, a tyto změny často vyvolají potřebu upravit datový model tak, aby co nejlépe reflektoval modelovanou realitu. Právě pro tuto činnost jsou dnes stále častěji používané CASE nástroje, které umožňují, nejen, snadnou změnu datového modelu prostřednictvím grafického uživatelského rozhraní bez nutnosti psaní vlastních DDL skriptů.

Vyvíjená aplikace slouží pro návrh a aktualizace schématu relační databáze, konkrétně podporuje DBMS Microsoft SQL Server a Oracle Database, z toho důvodu je první kapitola věnována popisu notace E-R diagramu, který se používá pro vizualizaci vztahů mezi entitami v relačním schématu nejčastěji. Další kapitola se zaměřuje na srovnání možností již existujících nástrojů pro tvorbu E-R diagramu. V následující kapitole je popsán návrh a implementace grafického editoru E-R diagramu, včetně algoritmů pro vyhledávání cest při vizualizaci vztahů a porovnání jejich výkonosti. Poslední kapitola se zabývá synchronizací mezi E-R diagramem a relačním schématem, zejména spoluprací použitých návrhových vzorů pro dosažení požadované funkcionality pro oba, dříve zmíněné, systémy řízení báze dat.

\section{Popis notace E-R diagramu}

\section{Popis a srovnání stávajících CASE nástrojů pro modelování relačních databází}

\section{Návrh a implementace grafického editoru E-R diagramů}

\section{Návrh a implementace synchronizace E-R diagramů s relačním schématem}

\section{Závěr}

\newpage

\begin{thebibliography}{99}
	\bibitem{ferda} Ferda Marvenec: Kdesi cosi.
\end{thebibliography}


\appendix
\section{Použité knihovny třetích stran}
\begin{itemize}
	\item MetroLib
	\item AvalonDock
\end{itemize}

\section{Struktura přiloženého optického média}
\textbf{Složka} Popis

\section{Instalace programu}
Popis instalace.

\end{document}
