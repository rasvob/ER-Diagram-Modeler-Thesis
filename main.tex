% Nejprve uvedeme tridu dokumentu s volbami
\documentclass[czech,bachelor,public,dept460,male,oneside]{diploma}
\usepackage{subfig}		% makra pro "podobrazky" a "podtabulky"


\ThesisAuthor{Radek Svoboda}

\CzechThesisTitle{Nástroj pro modelování relační databáze}

\EnglishThesisTitle{Relational Database Modeling Tool}

\SubmissionDate{28. dubna 2017}

\Thanks{Rád bych na tomto místě poděkoval vedoucímu mé bakalářské práce Ing. Petru Lukášovi za poskytnuté rady a čas strávený konzultacemi.}

\ThesisAssignmentImagePath{Figures/Assignment}

% Zadame soubor s digitalizovanou podobou prohlaseni autora zaverecne prace.
% Pokud toto makro zapoznamkujeme sazi se cisty text prohlaseni.

%\AuthorDeclarationImageFile{Figures/AuthorDeclaration.jpg}


\CzechAbstract{Tato bakalářská práce se zabývá návrhem a implementací softwarového nástroje pro návrh schématu relační databáze formou ER diagramu. Nástroj umožní uživateli nejen tvorbu nových schémat databáze, ale také vizualizaci již existujících schémat a jejich následné modifikace prostřednictvím změn ER diagramu v grafickém editoru. V první části se práce zabývá notací ER diagramu a srovnáním již existujících nástrojů určených k tvorbě ER diagramů. Druhá část se zaměřuje na návrh a implementaci jak grafického editoru, tak na problematiku synchronizace ER diagramu s existujícím relačním schématem.}

\CzechKeywords{relační databáze, ER diagram, relační schéma, CASE nástroj}

\EnglishAbstract{This thesis describes design and implementation of software tool for relational database schema design in form of an ER diagram. Tool is not limited only for creation of new database schemas, but it also provides visualization of already existing schemas and their subsequent modifications through changing ER diagram in graphic editor. The first part deals with ER diagram notation and the comparison of existing tools for the creation of ER diagrams. The second part focuses on the design and implementation of both the graphical editor, and on the synchronization of ER diagram with existing relational schema.}

\EnglishKeywords{relational database, ER diagram, relational schema, CASE tool}

\AddAcronym{SQL}{Structured Query Language}
\AddAcronym{DDL}{Data Definition Language}
\AddAcronym{CASE}{Computer-Aided Software Engineering}
\AddAcronym{ERD}{Entity-Relationship Diagram}
\AddAcronym{DBMS}{Database Management System}
\AddAcronym{DSD}{Data Structure Diagram}


\begin{document}

\MakeTitlePages

% Pokud mame v zaverecne praci vypisy kodu, jinak odstranit.
\lstlistoflistings

\section{Úvod}
Strukturované uložení dat je dnes jednou ze základních potřeb většiny vyvíjených aplikací. S rostoucím množstvím uchovávaných dat a potřebou jejich analýzy už dávno není dostačující prosté uložení do souboru. Na tyto potřeby reagují, dnes ve velké míře používané, relační databáze. Před samotným uložením dat je nutné definovat jejich strukturu, konkrétně relační model. Dnešní svět se velmi rychle vyvíjí, tedy mění, a tyto změny často vyvolají potřebu upravit datový model tak, aby co nejlépe reflektoval modelovanou realitu. Právě pro tuto činnost jsou dnes stále častěji používané CASE~nástroje, které umožňují, nejen, snadnou změnu datového modelu prostřednictvím grafického uživatelského rozhraní bez nutnosti psaní vlastních DDL skriptů.

Vyvíjená aplikace slouží pro návrh a aktualizace schématu relační databáze, konkrétně podporuje DBMS Microsoft SQL~Server a Oracle~Database, z toho důvodu je první kapitola věnována popisu notace ER diagramu, který se používá pro vizualizaci vztahů mezi entitami v relačním schématu nejčastěji. Další kapitola se zaměřuje na srovnání možností již existujících nástrojů pro tvorbu ER diagramu. V následující kapitole je popsán návrh a implementace grafického editoru ER diagramu, včetně algoritmů pro vyhledávání cest při vizualizaci vztahů a porovnání jejich výkonosti. Poslední kapitola se zabývá synchronizací mezi ER diagramem a relačním schématem, zejména spoluprací použitých návrhových vzorů pro dosažení požadované funkcionality pro oba, dříve zmíněné, systémy řízení báze dat.

% Zdroje:
% https://www.lucidchart.com/pages/er-diagrams
% http://dbedu.cs.vsb.cz/SubPages/OpenFile.aspx?file=book/dbcb.pdf
% https://en.wikipedia.org/wiki/Data_structure_diagram
% https://en.wikipedia.org/wiki/Data_dictionary
% https://www.lucidchart.com/pages/ER-diagram-symbols-and-meaning
\section{Popis notace ER diagramu}
V kontextu relačních databázi jsou ER diagramy nejběžnějším způsobem vizualizace entit vyskytujících se v databázi a jejich vzájemných vztahů. V průběhu let vznikla celá řada notací, které se liší nejčastěji použitými grafickými symboly pro znázornění jak entit, tak kardinality a povinnosti vztahů.

Dodnes nepředstavuje žádná z notací standard v oblasti modelování databází, což je nejvíce patrné na celé řadě CASE nástrojů, které používají mnohdy svou vlastní notaci či dokonce kombinaci symbolů z několika různých notací.

	\subsection{Historie}
	Za předchůdce ER diagramu jsou považovány diagramy datových struktur (DSD), které rovněž slouží k zachycení entit, vztahů mezi nimi včetně vztažených omezení. Cílem DSD je graficky znázornit dekompozici složitých entit na jednotlivé elementy, k zachycení struktury dat se předpokládalo použití tzv. datových slovníků, které se dají zjednodušeně popsat jako množiny tabulek obsahující metadata, z tohoto principu vychází i systémové katalogy moderních DBMS.
	
	\begin{figure}[!h]
		\centering
		\includegraphics[width=0.75\textwidth]{Figures/BachmanDiagram}
		\caption{Bachmanův diagram}
	\end{figure}
	
	Speciálním typem DSD je Bachmanův diagram, který slouží k vytvoření relačního datového modelu bez ohledu na způsob fyzického uložení dat v systému. Tento diagram měl patrně největší vliv na pozdější vznik ER diagramu, který publikoval Peter Chen roku 1976 \cite{chenERD}.
	
	Největším rozdílem mezi DSD a ERD, opomineme-li grafickou podobu, je fakt, že DSD se zaměřuje na vztahy mezi jednotlivými elementy entit, zatímco ERD řeší vztahy mezi samotnými entitami.
	
	\begin{figure}[!h]
		\centering
		\includegraphics[width=0.75\textwidth]{Figures/ChenVsDSD}
		\caption{Srovnání DSD a ERD (Chenova notace) \\ a) Data structure diagram b) ER diagram}
	\end{figure}
	
	\subsection{Využití ER diagramu}
	Nejčastější použití ER diagramu představuje tvorba nových relačních schémat při návrhu databází, ale rozhodně se nejedná o jedinou oblast využití. Tvorba ER diagramu velmi často stojí na počátku vývoje mnoha informačních systému, protože relační databáze jsou obvykle primárním datovým zdrojem těchto systému pro uložení dat transakčního charakteru. 
	
	ER diagramy umožňují zachytit vztahy a omezení entit v problémové doméně, proto jsou používány nejen při návrhu datové vrstvy, ale i v dalších fázích softwarového procesu. Doménový model aplikace je často ovlivněn právě ER diagramem relační databáze, protože na základě jejich odlišností jsme schopni odhadnout trivialitu objektově-relačního mapování. Trivialita tohoto mapování je často zásadní pro volbu vhodného návrhového vzoru pro implementaci vrstvy přístupu k datovému zdroji.
	
	Dalším velmi častým použitím ER diagramu je vizualizace již existujících schémat databáze, protože právě grafické znázornění schématu je ideální pro snadné odhalení logických chyb, které mají za následek omezené možnosti aplikace, v některých případech i propady výkonu při netriviálních dotazech pro získání dat. 
	Vizualizace existujících schémat nemusí být použitá pouze pro odhalení chyb, ale také při situaci, kdy došlo ke změně některých business procesu. Tyto změny často vyvolají nutnost přizpůsobit relační model změnám, ER diagram je tedy vhodný prostředek nejen k analýze stávajícího schématu pro návrh změn, ale také pro provádění těchto změn použitím CASE nástrojů.
	
	\subsection{Prvky ER diagramu}
	V této části jsou popsány prvky používané při tvorbě ER diagramu, včetně speciálních případů těchto prvků a příkladů. Ilustrační grafické znázornění odpovídá notaci podle P. Chena, nicméně v další části budou ostatní používané notace dále rozvedeny.
	
	\subsubsection{Entitní typ}
	Entitní typy (často označovány v souvislosti s ERD jen jako entity) jsou jednoznačně identifikovatelné objekty vyskytující se v problémové doméně, které jsou pro nás zajímavé svými atributy, daty, které chceme uchovávat např. \textit{zákazník, oddělení, automobil}. Obvykle se označují podstatným jménem v jednotném čísle. 
	
	\noindent Rozlišujeme následující typy entit:

	\begin{itemize}
		\item \textbf{Silný entitní typ} - muže existovat nezávisle na ostatních entitních typech, protože obsahuje alespoň jeden atribut, který entitu jednoznačně odlišuje od ostatních, značí se obdelníkem
		
		\item \textbf{Slabý entitní typ} - jeho existence je závislá na jiném entitním typu, protože neobsahuje atribut umožňující jednoznačnou identifikaci, značí se obdelníkem s dvojitým okrajem
		
		\item \textbf{Asociativní entitní typ} - 
	\end{itemize}
	
	\subsection{Rozdíly notací}
	
	\subsection{Nejčastěji používané notace} % TODO: Obrázky notací
	
	\subsection{Dědičnost} %TODO: Zachycení dědičnosti v ER - vzory
	
	\subsection{Omezení ER diagramu}
	
	\subsection{Zhodnocení}

\newpage
\section{Popis a srovnání stávajících CASE nástrojů pro modelování relačních databází}

\newpage
\section{Návrh a implementace grafického editoru ER diagramů}

\newpage
\section{Návrh a implementace synchronizace ER diagramů s relačním schématem}

\newpage
\section{Závěr}

\newpage

% TODO: Dopsat zdroje - vždy nad kapitolou
\begin{thebibliography}{99}
	\bibitem{chenERD} CHEN, Peter Pin-Shan. \textit{The entity-relationship model---toward a unified view of data. ACM Transactions on Database Systems} [online]. 1(1), 9-36 [cit. 2017-03-10]. Dostupné z: http://portal.acm.org/citation.cfm?doid=320434.320440
	
	\bibitem{whatIsERD}Lucidchart. \textit{What is an Entity Relationship Diagram} [online]. [cit. 2017-03-10]. Dostupné z: https://www.lucidchart.com/pages/er-diagrams
\end{thebibliography}


\appendix
% TODO: Linky na knihovny z GitHubu
\section{Použité knihovny třetích stran}
\begin{itemize}
	\item MetroLib
	\item AvalonDock
\end{itemize}

\section{Struktura přiloženého optického média}
\textbf{Složka} Popis

\section{Instalace programu}
Popis instalace.

\end{document}
